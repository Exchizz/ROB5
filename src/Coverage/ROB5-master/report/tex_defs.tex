\documentclass[11pt, twoside]{report}


% ------- Enable UTF8 characters ------- %
\usepackage[utf8]{inputenc}
\usepackage[english]{babel}


% ------- Page layout ------- %
\usepackage{fullpage}
\headsep = 24pt % spacing between header and text
\usepackage{fancyhdr}
\usepackage{amsmath}
\usepackage{paralist}
\fancyhf{}
\fancyhead[LE]{\slshape \rightmark} % section
\fancyhead[RE]{\thepage}
\fancyhead[RO]{\slshape \leftmark} % chapter
\fancyhead[LO]{\thepage}
\pagestyle{fancy}
\setlength{\headheight}{15pt}

\usepackage[toc,page]{appendix}

\usepackage{blindtext} % lorem ipsum replica

\usepackage{hyperref} % clickable references
\hypersetup{
    colorlinks,
    citecolor=black,
    filecolor=black,
    linkcolor=black,
    urlcolor=black
}
\usepackage{array}

%\newcolumntype{Le}[1]{%
%>{\raggedleft\hspace{0pt}}p{#1}}%

%\newcolumntype{Ri}[1]{%
%>{\raggedright\hspace{0pt}}p{#1}}%

%\newcommand{\tn}{\tabularnewline}


% ------- Math ------- %
\newcommand{\bm}[1]{\mbox{\boldmath $#1$}}	% bold math characters
\usepackage{icomma}							% use , as decimal pointer



%-------  Coding ------- %
\usepackage{listings}
\lstset{
	%backgroundcolor		= \color{anti-flashwhite},
	frame				= none,
 	language			= C++,
 	aboveskip			= 3mm,
 	belowskip			= 3mm,
 	showstringspaces	= false,
	columns				= flexible,
	basicstyle			= {\scriptsize\ttfamily},
	numbers				= left,
	numberstyle			= \tiny\color{gray},
	keywordstyle		= \color{blue},
	commentstyle		= \color{dkgreen},
	stringstyle			= \color{mauve},
 	breaklines			= true,
	breakatwhitespace	= true,
	tabsize				= 3,
	moredelim			= **[is][\color{mauve}]{@}{@},
}
%\lstinline|•|
%\begin{lstlisting}
%\lstin­put­list­ing{file­name.c}

\lstdefinelanguage{VHDL}{
  morekeywords={
    abs,access,after,alias,all,and,architecture,array,assert,attribute,
    begin,block,body,buffer,bus,
    case,component,configuration,constant,
    disconnect,downto,
    else,elsif,end,entity,exit,
    file,for,function,generate,generic,generic,
    if,impure,in,inertial,inout,is,
    label,library,linkage,literal,loop,
    map,mod,
    nand,new,next,nor,not,null,
    of,on,open,or,others,out,
    package,port,postponed,procedure,process,pure,
    range,record,register,reject,rem,report,return,rol,ror,
    select,severity,shared,shared,sla,sll,sra,srl,subtype,
    then,to,transport,type,
    unaffected,units,until,use,
    variable,
    wait,when,while,with,
    xnor,xor
  },
  morekeywords={rising_edge,falling_edge},
  sensitive=false,
  morecomment=[l]--
}
\lstdefinestyle{vhdl}{
  language     = VHDL,
  basicstyle   = \ttfamily,
  keywordstyle = \color{blue}\bfseries,
  commentstyle = \color{dkgreen}
}
\lstset{
	emph={falling_edge,rising_edge,std_logic_vector,std_logic,conv_integer}, emphstyle=\color{mauve}
}


% ------- Images ------- %
\usepackage{graphicx}
\usepackage{caption}
\usepackage{float}
\usepackage{subcaption}
\DeclareCaptionFont{gray}{\color{gray}}
\captionsetup{textfont={footnotesize,sc,gray},font={footnotesize,sc,gray}}
\usepackage{pdfpages}
% newcounter
\newcounter{includepdfpage}

% ------- Colors ------- %
\usepackage{color}
\definecolor{dkgreen}{rgb}{0,0.6,0}
\definecolor{gray}{rgb}{0.5,0.5,0.5}
\definecolor{mauve}{rgb}{0.58,0,0.82}
\definecolor{ashgrey}{rgb}{0.7, 0.75, 0.71}
\definecolor{anti-flashwhite}{rgb}{0.95, 0.95, 0.96}


% ------- TiKz ------- %
\usepackage{tikz}
\usetikzlibrary{shapes.multipart}
\usetikzlibrary{automata,arrows}
\usetikzlibrary{shapes,snakes}

\tikzset{
    data/.style={
        draw,
        rectangle split,
        rectangle split parts=4,
        text centered,
    },
    data+/.style={
        data,
        rectangle split every empty part={},% resets empty-part macro (explanation below)
        rectangle split empty part width=\widthof{#1},
        rectangle split empty part height=\heightof{#1},
        rectangle split empty part depth=\depthof{#1},
    },
}


\makeatletter
\pgfdeclareshape{datastore}{
  \inheritsavedanchors[from=rectangle]
  \inheritanchorborder[from=rectangle]
  \inheritanchor[from=rectangle]{center}
  \inheritanchor[from=rectangle]{base}
  \inheritanchor[from=rectangle]{north}
  \inheritanchor[from=rectangle]{north east}
  \inheritanchor[from=rectangle]{east}
  \inheritanchor[from=rectangle]{south east}
  \inheritanchor[from=rectangle]{south}
  \inheritanchor[from=rectangle]{south west}
  \inheritanchor[from=rectangle]{west}
  \inheritanchor[from=rectangle]{north west}
  \backgroundpath{
    %  store lower right in xa/ya and upper right in xb/yb
    \southwest \pgf@xa=\pgf@x \pgf@ya=\pgf@y
    \northeast \pgf@xb=\pgf@x \pgf@yb=\pgf@y
    \pgfpathmoveto{\pgfpoint{\pgf@xa}{\pgf@ya}}
    \pgfpathlineto{\pgfpoint{\pgf@xb}{\pgf@ya}}
    \pgfpathmoveto{\pgfpoint{\pgf@xa}{\pgf@yb}}
    \pgfpathlineto{\pgfpoint{\pgf@xb}{\pgf@yb}}
 }
}
\makeatother

\newcounter{wavenum}
 
\setlength{\unitlength}{1cm}
% advance clock one cycle, not to be called directly
\newcommand*{\clki}{
 \draw (t_cur) -- ++(0,.3) -- ++(.5,0) -- ++(0,-.6) -- ++(.5,0) -- ++(0,.3)
   node[time] (t_cur) {};
}
 
\newcommand*{\bitvector}[3]{
 \draw[fill=#3] (t_cur) -- ++( .1, .3) -- ++(#2-.2,0) -- ++(.1, -.3)
                        -- ++(-.1,-.3) -- ++(.2-#2,0) -- cycle;
 \path (t_cur) -- node[anchor=mid] {#1} ++(#2,0) node[time] (t_cur) {};
}
 
% \known{val}{length}
\newcommand*{\known}[2]{
   \bitvector{#1}{#2}{white}
}
 
% \unknown{length}
\newcommand*{\unknown}[2][XXX]{
   \bitvector{#1}{#2}{black!20}
}
 
% \bit{1 or 0}{length}
\newcommand*{\bit}[2]{
 \draw (t_cur) -- ++(0,.6*#1-.3) -- ++(#2,0) -- ++(0,.3-.6*#1)
   node[time] (t_cur) {};
}
 
% \unknownbit{length}
\newcommand*{\unknownbit}[1]{
 \draw[ultra thick,black!50] (t_cur) -- ++(#1,0) node[time] (t_cur) {};
}
 
% \nextwave{name}
\newcommand{\nextwave}[1]{
 \path (0,\value{wavenum}) node[left] {#1} node[time] (t_cur) {};
 \addtocounter{wavenum}{-1}
}
 
% \clk{name}{period}
\newcommand{\clk}[2]{
   \nextwave{#1}
   \FPeval{\res}{(\wavewidth+1)/#2}
   \FPeval{\reshalf}{#2/2}
   \foreach \t in {1,2,...,\res}{
       \bit{\reshalf}{1}
       \bit{\reshalf}{0}
   }
}
 
% \begin{wave}[clkname]{num_waves}{clock_cycles}
\newenvironment{wave}[3][SSI\_CLK]{
 \begin{tikzpicture}[draw=black, yscale=.7,xscale=1]
   \tikzstyle{time}=[coordinate]
   \setlength{\unitlength}{1cm}
   \def\wavewidth{#3}
   \setcounter{wavenum}{0}
   \nextwave{#1}
   \foreach \t in {0,1,...,\wavewidth}{
     \draw[dotted] (t_cur) +(0,.5) node[above] {t=\t} -- ++(0,.4-#2);
     \clki
   }
}{\end{tikzpicture}}
 
\usepackage{verbatim}

% ------- Comments & ToDo's ------- %
\usepackage{todonotes}
\usepackage{comment}

\usepackage{parskip}
\setlength{\parindent}{0pt}



\usepackage{epstopdf}