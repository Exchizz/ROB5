\section{Planning}
\label{sec::planning}
All rooms in the map must be checked in order to find cups. The robot must be within 2 meters of a cup in order to actually detect the cup and within 1 meter in order to collect it. Cups are marked in the map using one pixel with grayscale value 150. Cups can be unloaded at the two offloading stations in the cantina. The offloading stations are represented with pixel values 100. The robot must start and end at an offloading station.\\[0.2cm]
You are free in regard in choice of algorithms. However, please document what algorithm you choose, how many kilometres the robot moves and how long it takes to calculate the path the robot takes.\\[0.2cm]
All planning is done offline, and it involves the functions:
\begin{itemize}\itemsep-3pt
\item Offline Wavefront
\item 
\end{itemize}

\subsection{Method}
Offline wavefront

To control the order of the rooms to be cleared, has been made a graph consisting of sub--graphs, one for each block. Each room in a blok is connected to the center of the block, and each center of the blocks are connected to the closest unloading--station. To detect a room, first the upper left corners of the squares are detected, then the upper right and then the lower left. The last corner is computed, assuming that is , and the pair positions for each corner is stored in a data type, called square.\\[0.2cm]
To detect each pixel, 

\subsection{Results}
How long does it take?

\subsection{Conclusion}
What works and what does not? Why?
